\documentclass{article}
\usepackage[margin=1.0in]{geometry}
\usepackage[utf8]{inputenc}
\usepackage[english]{babel}
\usepackage[
backend=biber,
style=numeric,
sorting=ynt
]{biblatex}
\bibdata{ref.bib}

\begin{document}


\title{Immunova identifies immunophenotypes in acute peritonitis}
\date{10-12-2019}
\author{Ross Jake Burton}

\pagenumbering{gobble}
\maketitle
\newpage
\pagenumbering{arabic}

\begin{flushleft}
\section{Immunova provides a framework for reproducible autonomous cytometry analysis}
	
Autonomous cytometry analysis can be broadly divided into two categories: automated gating 	
and high dimensional clustering. The former seeks to replicate the traditional approach of subsetting single cell data into 'populations' by encircling data in hand-drawn polygons in two-dimensional space. The feature space is explored in sequential bi-axial plots until the user has isolated the population of interest. The ordering of these sequential plots is commonly termed the 'gating strategy'. When refering to automated gating, we emply that clustering algorithms are applied in two-dimensional space, replicating the human generate polygons, and follow the order of some pre-defined 'gating strategy'. This is the methodology employed by \cite{opencyto}.
\end{flushleft}



\end{document}